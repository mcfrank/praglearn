\documentclass{article} % For LaTeX2e
\usepackage{nips13submit_e,times}
\usepackage{hyperref}
\usepackage{url}
\usepackage{amsmath}
\usepackage{graphicx}

\title{Learning words through recursive pragmatic reasoning about other agents}

\author{
Nathaniel Smith\thanks{...} \\
School of Informatics\\
University of Edinburgh\\
Edinburgh, Scotland, UK\\
\texttt{nathaniel.smith@ed.ac.uk} \\
\AND
Noah D. Goodman \\
Department of Psychology\\
Stanford University \\
Stanford, CA, USA \\
\texttt{ngoodman@stanford.edu} \\
\And
Michael C. Frank \\
Department of Psychology \\
Stanford University \\
Stanford, CA, USA \\
\texttt{mcfrank@stanford.edu}}

\newcommand{\fix}{\marginpar{FIX}}
\newcommand{\new}{\marginpar{NEW}}

% \nipsfinalcopy % Uncomment for camera-ready version

\begin{document}

\maketitle

\begin{abstract}
  Adult language users are remarkably good at making inferences about speakers' intentions in context, and children learning their native language also display substantial pragmatic skill in acquiring the meanings of unknown words. While pragmatic inference and word learning have both been characterized in probabilistic terms, no current models make appropriate inferences about the meanings of words in pragmatically-rich inferential contexts. We describe a model in which language learners assume that they jointly approximate a shared, external lexicon and reason recursively about the goals of others in using this lexicon. This model captures phenomena in word learning and pragmatic inference; it additionally leads to insights about the emergence of communicative systems in conversation and the mechanisms by which pragmatic inferences become incorporated into word meanings.
\end{abstract}

\section{Introduction}

Natural language is a powerful tool for conveying meaning in context. Our utterances are pervasively underspecified. Consider ``it's raining,'' ``I ate some of the cookies,'' or ``can you close the window?'' In each, a listener must go beyond the literal meaning of the words to fill in contextual details (``it's raining here and now''), infer that a stronger alternative is not true (``I ate some but not all of the cookies''), or more generally infer the speaker's communicative goal (``I want you to close the window right now because I'm cold.'').

Following this intuition, theories of pragmatic communication frame the process of language comprehension as inference about the generating goal of an utterance given a rational speaker \cite{grice1975,dale1995,frank2012}. For example, a listener might reason, ``if she had wanted me to think `all' of the cookies, she would have said `all'---but she didn't. Hence `all' must not be true and she must have eaten some {\it but not all} of the cookies.'' This kind of counterfactual reasoning has a recursive character to it: listener models speaker (who may in turn be modeling a listener).

Even before they are proficient language users, children learn the meanings of words from their pattern of use in context. For example, in demonstrations of the ``disambiguation effect,'' a novel word (e.g. ``dax'') is used in a context containing both a novel and a familiar object. From quite young, children tend to make the inference that ``dax'' refers to the novel object \cite{markman1988}. In another demonstration, even young toddlers can extract mappings between words and objects from a pattern of individually-ambiguous co-occurrences between pairs of words and pairs of objects \cite{smith2008}. Taking inspiration from children's learning prowess, recent work in the field of grounded language learning has explored the possibility that even relatively complex word meanings can be inferred from corpus data (e.g. \cite{zettlemoyer2005,chen2008,frank2009,kwiatkowski2010,johnson2012}).

Recursive pragmatic inferences such as those described above could be an important part of language learning.\footnote{Very young children make inferences that are often labeled as ``pragmatic'' in that they involve reasoning about context \cite{clark1988,baldwin1993}, though they also show weaknesses in specific inferences (e.g. strengthening ``some'' to ``some but not all'' \cite{papafragou2003}). Here we remain agnostic about the age at which children are able to make such inferences robustly, as it may vary depending on the linguistic materials being used in the inference \cite{barner2011}.} For example, the disambiguation inference described above can emerge from such reasoning (``why would the speaker have said ``dax'' if there was a word for the known object that she knew I would know'') \cite{clark1988}. In addition, there are a variety of intriguing cases in which listeners seem to create situation- and task-specific ways of referring to particular objects. For example, when asked to refer to idiosyncratic geometric shapes, over the course of an experimental session, participants create conventionalized descriptions that allow them to perform accurately even though they do not begin with a shared label \cite{krauss1964,clark1986}. Yet models of grounded language learning have failed to take pragmatic factors into account.

The goal of our current work is to investigate the possibilities for integrating models of recursive pragmatic reasoning with models of language learning, with the hope of capturing phenomena in both domains. We begin by describing previous models in each domain as well as proposals for bringing the two together. We next simulate findings from the domains of pragmatic inference, word learning, and their intersection (summarized in Table \ref{tab:results}). We end by discussing the implications of our results for work on language change.

\section{Models}

Basic idea of iterated pragmatic reasoning models. $S_i$ defines a causal model for utterances described by a simple Bayes net. $L_{i+1}$ does standard rational inference over this network. But the likelihood function in $S_i$'s model is very complicated, and includes equations derived from considering how a hypothetical $L_{i-1}$ would interpret utterances if they were doing rational inference about another hypothetical causal agent $S_{i-2}$.

To simplify matters we envision a simple world that has $k$ words and $l$ objects. The \textit{lexicon} defines the literal meaning of each word, ...

extensional as marginalization over intensional

\newcommand{\word}{\text{word}}
\newcommand{\obj}{\text{object}}
\begin{align*}
P_{L_0}(\obj | \word) &= \text{lexicon}(\word, \obj) P_{\text{prior}}(\obj | \text{context}). \\
P_{S_i}(\word | \obj) &\propto \exp\Big(\lambda \big(\log P_{L_{i - 1}}(\obj | \word) - \text{cost}(\word)\big)\Big) \\
P_{L_i}(\obj | \word) &\propto P_{S_{i-1}}(\word | \obj) P_{\text{prior}}(\obj | \text{context})
\end{align*}

For the rest of this paper we somewhat arbitrarily fix the softmax parameter to $\lambda = 3$.

\subsection{Naive literal learner model}

To see how
Make $L_0$ uncertain about values in $\text{lexicon}(\word, \obj)$, impose a prior, and then

Needs feedback

can't switch roles from speaker to listener

knocks out the lexical uncertainty/horn implicature results (because you can only have uncertainty at one level)

something wrong about taking evidence from the pragmatic speaker and using it to train the literal listener

\subsection{``Social anxiety'' model}

\section{Experiments}

\begin{table}[t]
\caption{Table of empirical results and references}
\label{tab:results}
\begin{center}
\begin{tabular}{llllll}
\hline \\
Phenomenon & Ref. & Word learning \cite{frank2009} & Pragmatics \cite{frank2012,bergen2012}  & Model 1 & Model 2 \\
\hline
Scalar implicature & \cite{stiller2011} & & & & \\
Horn implicature & & & & & \\
Cross-situational learning & \cite{yu2007} & & & & \\
Disambiguation & \cite{markman1988} & & & & \\
Communicative equilibria & \cite{galantucci2007} & & & & \\
\end{tabular}
\end{center}
\end{table}

\subsection{Pragmatic reasoning}

\subsubsection{Scalar implicature}

\subsubsection{Horn implicature}

\subsection{Word learning}

\subsubsection{Statistical cross-situational word learning}

\subsubsection{Disambiguation using known words}

\subsection{Lexical learning in pragmatic contexts}

\subsubsection{Emergence of communicative equilibria}

\begin{figure}
\centering
\includegraphics[width=0.48\textwidth]{figures/emergence2x2-1.pdf}
\includegraphics[width=0.48\textwidth]{figures/emergence2x2-3.pdf} \\
\includegraphics[width=0.48\textwidth]{figures/emergence3x3-0.pdf}
\includegraphics[width=0.48\textwidth]{figures/emergence3x3-1.pdf} \\
\caption{\label{fig:emergence} Simulations....}
\end{figure}


\subsubsection{Lexicalization of Horn implicatures}

\begin{figure}
\centering
\includegraphics[width=0.48\textwidth]{figures/horn-emergence-0.pdf}
\includegraphics[width=0.48\textwidth]{figures/horn-emergence-1.pdf}
\caption{\label{fig:horn} Simulations.}
\end{figure}


\subsubsection{Lexicaliation and preservation of scalar implicature}


\begin{figure}
\centering
\includegraphics[width=2in]{figures/some-all-only-pragmatic.pdf}
\includegraphics[width=2in]{figures/some-all-pragmatic+unambiguous.pdf}
\caption{\label{fig:scalar} Simulations..}
\end{figure}

\section{Conclusion}

%\subsubsection*{Acknowledgments}


% Thanks to ONR Grant #blahblah


\bibliographystyle{plain}
\bibliography{pragmatics}

\end{document}
